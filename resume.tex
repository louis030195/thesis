%%%%%%%%%%%%%%%%%%%%%%%%%%%%%%%%%%%%%%%%%%%%%%%%%%%%%%%%%%%%%%%%%%%%%%%%%%%%%%%%%
% Résumé en français
%%%%%%%%%%%%%%%%%%%%%%%%%%%%%%%%%%%%%%%%%%%%%%%%%%%%%%%%%%%%%%%%%%%%%%%%%%%%%%%%%

\chapter{Résumé}
Dans ce travail, je présente une solution hébergée dans le cloud facile à déployer, à maintenir, à mettre à l'échelle et à tarifer pour la surveillance de la faune à l'aide de l'intelligence artificielle, qui permet d'effectuer une vaste collecte et analyse de données et de déployer et suivre rapidement de nouvelles données de surveillance pour tout groupe particulier d'espèces et dans tout habitat. En particulier, je montre comment ce logiciel peut évoluer de quelques images initiales à des dizaines de milliers de vidéos par mois.

La surveillance de l'état des espèces de notre planète à l'aide d'une grande variété de technologies et de plateformes est cruciale pour améliorer la conservation de la biodiversité et pour développer de nouvelles méthodes de gestion des espèces. Je montre comment mettre en œuvre une application de surveillance de la faune directement dans le cloud à l'aide de l'intelligence artificielle (IA). Cette solution offre également des moyens d'analyser les données pour la surveillance des espèces et la capacité d'agréger les données afin d'identifier les modèles et les tendances pour informer les écologistes.

Plus important encore, pour l'identification des animaux, le système permet d'économiser un temps considérable de travail manuel tout en offrant le même niveau de précision que les bénévoles humains.