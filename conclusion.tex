%%%%%%%%%%%%%%%%%%%%%%%%%%%%%%%%%%%%%%%%%%%%%%%%%%%%%%%%%%%%%%%%%%%%%%%%%%%%%%%%%
% Conclusion
%%%%%%%%%%%%%%%%%%%%%%%%%%%%%%%%%%%%%%%%%%%%%%%%%%%%%%%%%%%%%%%%%%%%%%%%%%%%%%%%%

\chapter{Conclusion}
%% La conclusion est la partie dans laquelle l’étudiant tire les enseignements de son travail et de son raisonnement, en répondant à sa question de recherche et éventuellement en vérifiant ou non son hypothèse. La conclusion permet également à l’étudiant de mettre son travail en perspective et de prendre de la hauteur par rapport à la problématique traitée.


In this work I presented an easy to deploy, maintain, scale and price cloud solution for wildlife monitoring powered by artificial intelligence.

First, an introduction to the context and who is concerned has been detailed.

Secondly, a quick explanation of the domain of computer vision, what makes it work and its use has been described.

The previous research in this field on the algorithmic and data side, in addition to the different tasks and its potential on wildlife monitoring has been reported.

Then, we tested and showed the results of the publicly accessible solutions, their disadvantages.
We developed and tested the considered solutions, proposed incremental concrete cloud pipelines that presented functional results, their price and time consumption and their ability to scale.

A link with ecological research has been established, be it about the possibility to predict species decline or growth and automatically detect anomaly in these populations.

Finally, the future of this solution, with the complex but long-term possible new tasks of computer vision has been expressed.

I hope that this document and the concrete cloud computer vision code will be used to save what is left of biodiversity on Earth, help those that are timid with cloud providers and contribute to the immensely important field of artificial intelligence.